\documentclass[a4paper,english,11pt,twoside]{article}
\usepackage[utf8]{inputenc}
\usepackage[T1]{fontenc}
\usepackage[english]{babel}
\usepackage{epsfig}
\usepackage{graphicx}
\usepackage{amsmath}
\usepackage{pstricks}
\usepackage{subfigure}
\usepackage{booktabs}
\usepackage{float}
\usepackage{gensymb}
\usepackage{preamble}
\restylefloat{table}
\renewcommand{\arraystretch}{1.5}
 \newcommand{\tab}{\hspace*{2em}}


\date{\today}
\title{Mandatory Assignment 1}
\author{Jørgen D. Tyvand}

\begin{document}
\maketitle
\newpage

\section*{Exercise 1}
We consider the ODE problem\\
\\
$u'' + \omega^2u = f(t),\tab u(0) = I, u'(0) = V, t \in (0,T]$\\
\section*{a)}
The first task is to derive the discretized equation for the first time step $(u^1).$\\
We start by discretizing the general problem using central difference:\\
\\
$\frac{u^{n+1} - 2u^n + u^{n-1}}{\Delta t^2} + \omega^2u^n = f$
$u^{n+1} = (2 - \omega\Delta t^2)u^n - u^{n-1} + \Delta t^2f$\\
\\
We use a central difference for the initial condition:\\
\\
$\frac{u^1 - u^{-1}}{2\Delta t} = V$
$u^{-1} = u^1 - 2\Delta t V$\\
\\
Inserted into the scheme for n = 0:\\
\\
$u^1 = (2 - \omega\Delta t^2)u^0 - u^1 + 2\Delta t V + \Delta t^2f$\\
\\
Giving, after a quick rearrangement:\\
\\
$u^1 = (1 - \frac{\omega\Delta t^2}{2})u^0  + \Delta t V +\frac{ \Delta t^2f}{2}$\\
\newpage
\section*{b)}
We are given the manufactured solution $u_e(x,t) = ct + d.$\\
Using the initial conditions of the ODE problem, we find the restrictions on c and d:\\
\\
$u_e(0) = d = I\tab u'_e(0) = c = V$\\
\\
We then find f by inserting $u_e$ into the ODE problem, noting that $u''_e = 0$:\\
\\
$f = \omega^2(ct + d) = \omega^2(Vt + I)$\\
\\
We then want to show that $[D_tD_t t]^n = 0$, so we use the central differencing scheme employed earlier, and insert for t:\\
\\
$\frac{t^{n+1} - 2t^n + t^{n-1}}{\Delta t^2} = \frac{(n+1)\Delta t - 2n\Delta t + (n-1)\Delta t}{\Delta t^2} = \frac{n\Delta t - \Delta t - 2n\Delta t + n\Delta t - \Delta t}{\Delta t^2} = \frac{2n - 2n}{\Delta t^2} = 0$\\
\\
We finally want to show that $u_e$ is a perfect solution, so we use the fact given in the exercise, $[D_tD_t(ct+d)]^n = 0$ and get:\\
\\
$[D_tD_t(ct+d)]^n + \omega^2(ct + d) = f^n$\\
$0 + \omega^2(Vt + I) = \omega^2(Vt+I)$\\
\\
So we see that $u_e$ is a perfect solution of the discrete equations.

\section*{c)}
See python program vib\_undamped\_verify\_mms.py

\section*{d)}
 See python program vib\_undamped\_verify\_mms.py

\section*{e)}
As we see from the function cubic in the program listed above, we get a residue for the first term of $c\Delta t^3$, where c is the coefficient of the third-degree term.\\
\\
A third degree polynomial will therefore not fulfill the discrete equations.

\section*{f)}
 See python program vib\_undamped\_verify\_mms.py

\section*{g)}
 See python program vib\_undamped\_verify\_mms.py
\newpage
\section*{Exercise 21}
\section*{a)}
See python program elastic\_pendulum.py
\section*{b)}
See python program elastic\_pendulum.py
\section*{c)}
For pure vertical motion, we can simplify the ODE problem:\\
\\
$\ppder{x}{t} = 0$\\
\\
$\ppder{y}{t} = -\frac{\beta}{1 - \beta}(1 - \frac{\beta}{\sqrt{(y - 1)^2}} - \beta$\\
\\
From Wolfram Alpha, $\frac{y-1}{\sqrt{(y - 1)^2}}$ is the sign of $(y-1)$. Here y must be less than 1, since the pendulum is fixed at scaled location y = 1. We therefore get:\\
\\
$\ppder{y}{t} = -\frac{\beta}{1 - \beta}(y - 1 + \beta) - \beta = -\frac{\beta y}{1 - \beta} + \frac{\beta}{1 - \beta} - \frac{\beta^2}{1-\beta} - \beta = -\frac{\beta y + \beta - \beta^2 - \beta(1 - \beta)}{1 - \beta} = -\frac{\beta}{1 - \beta}y = -\omega^2y$\\
\\
where $\omega = \sqrt(\frac{\beta}{1-\beta})$\\
\\
We also get:\\
\\
$y(0) = \epsilon\tab y(t) = \epsilon cos(\omega t)$
\section*{d)}
See python program elastic\_pendulum.py
 \end{document}
